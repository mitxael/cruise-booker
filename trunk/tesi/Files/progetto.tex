\chapter{Implementazione}


\section{MatrixPro e Trackla2}

MatrixPro {[}1{]}  un software con licenza GPL2, ed  quindi gratuito
e open source. Il tool  stato creato per permettere a studenti, ma
soprattutto a insegnanti di creare velocemente animazioni personalizzate
su strutture dati preesistenti nel programma. L'applicazione  molto
utile come spiegano gli sviluppatori nell'articolo {[}1{]} per rendere
pi coinvolgenti le lezioni universitarie, con presentazioni animate
precedentemente create dal docente, o addirittura create al volo durante
la lezione, per riuscire a rispondere con la dimostrazione visiva
a domande e dubbi degli studenti.

Tra i punti di forza di questo applicativo troviamo la semplicit
con cui possono essere utilizzate le strutture dati esistenti, ma
soprattutto l'organizzazione di queste. Le strutture sono suddivise
in due grandi gruppi: 
\begin{description}
\item [{FDT}] (Fundamental Data Types) che equivalgono alle strutture basilari,
primitive alle quali non viene associata alcuna operazione o informazione
sematica riguardo al loro utilizzo. Tra queste troviamo liste, alberi,
grafi, array, chiavi.
\item [{CDT}] (Conceptual Data Types) sono strutture che implementano tipi
di dato astratti che necessitano di maggiori restrizioni, e possono
essere modificate solo tramite operazioni che ne modificano la struttura
secondo regole predefinite. Tra queste troviamo Alberi Binari di Ricerca,
Heap, Stack, e molte altre.
\end{description}
Esistono inoltre altre strutture utilizzabili nella realizzazione
di un'animazione, che vengono chiamate {}``strutture utili'', e
permettono di creare al volo un input random o consentono altre funzionalit
allo stesso modo interessanti. Questa classificazione in strutture
primitive e avanzate aiuta l'utente a comprendere meglio la gerachia
esistente tra i diversi sistemi, e a concentrarsi maggiormente sulla
parte concettuale del problema analizzato e non sui dettagli implementativi.


\section{modifiche apportate}


\subsection{Esercizi + anche solo simulazione}


\subsection{Input a scelta}


\subsection{Documentazione algoritmo laterale}


\subsection{Domande relative esercizio}


\subsection{testo di aiuto}


\section{Esempi}


\section{Implementazione}

uno schema veloce delle classi, magari anche un activity con le operazioni
base che pu fare lo studente