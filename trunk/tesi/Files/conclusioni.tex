%
\chapter{Conclusioni}


\section{Sviluppi Futuri}

Il software  gi stato messo a disposizione degli studenti ed  scaricabile
alla pagina http://disi.unitn.it/\textasciitilde{}montreso/. Esso
fornisce un valido supporto alle lezioni ed un'integrazione al libro
di testo del corso.

Il programma  ancora sperimentale, il software risultante  ricco
di nuove funzioni che aumentano la partecipazione dell'utente nell'attivit,
ma che purtroppo non sono ancora perfette.

Come si  gi accennato nel capitolo precedente, le migliorie introdotte
possono essere ulteriormente perfezionate, per fornire agli studenti
un valido aiuto per lo studio. Nonostante non siano state ancora realizzate
nel progetto, sono gi state concepite pianificate alcune modifiche:
\begin{description}
\item [{Domande}] Innanzi tutto si potrebbe aumentare la quantit di domande
per ogni esercizio, e migliorare il sistema di inserimento delle risposte:
ad oggi il sistema giudica diverse le riposte {}``O(nlogn)'' e {}``O(n
logn)''. Come  gi stato accennato in \ref{sub:Domande-relative-esercizio},
sarebbe utile spostare le domande in un file di testo esterno, di
modo che non sia necessario ri-compilare tutto il codice del programma
nel caso si modifichi solo una domanda, e per permettere ad un docente
interessato di personalizzare le domande in accordo con il proprio
programma di studio.
\item [{Input1}] Nella sezione \ref{sub:Input-a-scelta}, si parla della
possibilit si fornire una terza tipologia di input, i {}``Test Cases'',
ovvero un insieme di dati preparati ad hoc per ogni algoritmo, per
evidenziare particolari comportamenti e caratteristiche del problema
in analisi. Sicuramente questa potrebbe essere una delle prime integrazioni
al programma della tesi. Un ulteriore miglioria possibile attuabile,
poterbbe consistere nel permettere di memorizzare in un file di testo
i test case per ogni esercizio, che sarebbero modificabili da un docente,
come per domande e documentazione
\item [{Input2}] La tecnica di inserimento dei dati potrebbe inoltre essere
integrata con il sistema di creazione di animazioni {}``al volo''
di MatrixPro. Il software, come spiegato in \ref{sec:MatrixPro-e-Trackla2},
permette di inserire e personalizzare strutture dati preinstallate
nell'applicazione per poter creare animazioni. Sarebbe un grande passo
avanti permettere agli utenti del programma di inserire i dati di
input di un esercizio con lo stesso sistema di creazione della animazioni
di MatrixPro: sicuramente renderebbe gli studenti pi fantasiosi nel
creare casi limite, soprattutto con strutture grafiche come grafi
e alberi.
\item [{Esercizi}] Il programma di un corso di Algoritmi  sicuramente
vasto, e per ogni argomento una dimostrazione non basta di sicuro.
Potrebbe essere necessario in futuro allargare la raccolta di esercizi
per fornire agli studenti un percorso ricco e completo di supporto
al proprio corso di studi. 
\end{description}
Al momento non  possibile permettere ad uno studente o ad un docente
di realizzare nuovi esercizi in maniera semplice e veloce, come invece
MatrixPro permette per la creazione di nuove animazioni. Sicuramente
questo potrebbe essere un buon punto di partenza per un successivo
lavoro al riguardo.


\section{Conclusioni}

da fare


\section{Ringraziamenti}

Desidero innanzitutto ringraziare il Professor Montresor per i preziosi
insegnamenti durante gli anni di laurea triennale e per le numerose
ore dedicate alla mia tesi. Intendo poi ringraziare la Helsinky University
of Technology per avermi fornito il software MatrixPro, indispensabile
per la realizzazione del progetto. Inoltre, vorrei esprimere la mia
sincera gratitudine ai miei compagni di corso, in particolare Stefano
e Alberto per i numerosi consigli durante la realizzazione del lavoro.
Infine, ho desiderio di ringraziare con affetto i miei genitori e
la mia famiglia per il sostegno ed il grande aiuto che mi hanno dato
ed in particolare Stefano per essermi stato vicino ogni momento durante
questo anno di lavoro.