\chapter{L'ulteriore azione accertatrice}
\section{Condizioni e limiti}
Nel capitolo precedente si è tentato di esaminare i caratteri dell'accertamento parziale. Questo capitolo ha l'obiettivo l'analisi delle attività che seguono all'emanazione del parziale fino alla scadenza del termine di decadenza sancito per la notifica di avvisi di accertamento riguardanti un dato periodo d'imposta.

Di seguito si affronteranno due problemi: il primo riguarda la necessità o l'eventualità di un ulteriore accertamento rispetto a quello parziale, l'altro attiene all'individuazione delle condizioni necessarie perchè possa essere emesso un ulteriore avviso di accertamento.

Riguardo al primo problema, già si è rilevato come l'art. 41 bis autorizzi l'Ufficio a proseguire l'ulteriore azione acertatrice senza precisare a quali condizioni ciò debba avvenire.
Andando più a fondo nella lettura dell'articolo, il legislatore prevede che i competenti uffici possono limitarsi ad accertare il reddito o il maggior reddito imponibile non dichiarato ovvero la maggiore imposta da versare. Nella previsione di tale facoltà, bisogna anche esaminare quali siano i presupposti ed i condizionamenti che portano l'amministrazione finanziaria a porre in essere un accertamento parziale. Tra i presupposti, si è visto che il valore altamente indicativo dell'elemento comunicato dalle fonti esterne assume un ruolo decisivo. Ciò posto, la valutazione dell'Ufficio si basa essenzialmente sul contenuto della segnalazione piuttosto che sulla situazione complessiva del contribuente. Aggiungendo questa affermazione al fatto che l'emanazione del parziale non preclude l'ulteriore azione accertatrice, si può supporre che l'Ufficio possa ricorrere alla procedura di cui all'art. 41 bis anche se ha già riscontrato l'esistenza di elementi diversi rispetto a quelli segnalati dalle fonti esterne.

Se si tengono in considerazione i principi di economia e di efficienza che regolano l'attività amministrativa, è facile ritenere che l'Ufficio non procederà all'emissione del parziale in tutti i casi in cui l'elemento che potrebbe essere posto a base dell'accertamento parziale risulti essere collocato in un contesto di elementi e circostanze note, perchè in tale evenienza esso avvierà le attività istruttorie ritenute necessarie per poter emanare in seguito un atto ordinario. Da tale ricostruzione si può dedurre che la conoscenza di diversi elementi a carico del contribuente non influisce sull'emanazione del parziale. Detta emanazione, però, potrebbe anche essere abbinata alla carenza di ulteriori elementi che potrebbero formare il presupposto per avviare le attività istruttorie.

A questo punto sembra lecito affermare che il semplice fatto di aver emanato un avviso di accertamento parziale non implica per l'Ufficio nessun obbligo di approfondire il controllo a carico del contribuente e la scelta di proseguire l'ulteriore azione accertatrice è rimessa alla valutazione dell'Ufficio stesso. Quindi, la prosecuzione dell'attività accertatrice non risulta nè impedita e nè necessitata dal parziale.

Prima di chiudere il discorso è necessario fare una precisazione. Il ragionamento fatto non deve indurre a ritenere che l'accertamento parziale sia una sorta di accertamento anticipato o provvisorio ovvero un accertamento con riserva di ampliamenti. Tale istituto, infatti, rappresenta uno dei modi in cui si esteriorizzano le pretese del Fisco e che possono esaurire le attività istruttorie ovvero coesistere con ulteriori sviluppi. Ciò vuol dire che una dichiarazione può essere oggetto di liquidazione ex art. 36 bis, ovvero di accertamento parziale, o di accertamento ordinario ma anche di tutte e tre le attività.

Avendo affrontato il primo problema in modo esaustivo, si può procedere con l'individuazione dei requisiti che devono sussistere perchè l'Ufficio possa legittimamente emettere un secondo avviso di accertamento.