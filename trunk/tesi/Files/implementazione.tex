
\chapter{Implementazione} 

In questo capitolo si descrivono e si motivano le scelte progettuali
realizzate nell'implementazione dell'applicazione, soffermandosi in
particolar modo sulle migliorie apportate al programma di partenza.
Sono inoltre descritti due esercizi rappresentativi \ref{sec:Esempi},
e una breve spiegazione sull'utilizzo da parte dello studente\ref{sec:Come-si-usa}.


\section{Descrizione iniziale}

Il software realizzato si presenta allo studente con una schermata
iniziale contenente l'elenco degli esercizi preparati organizzati
tramite una struttura ad albero per argomenti. Selezionando un particolare
esercizio si visualizza nella parte inferiore della finestra una breve
descrizione dell'esercitazione.

Una volta scelta l'attivit, allo studente si apre un'ulteriore finestra,
molto simile a quella del sistema Trakla2, ma con qualche integrazione,
di cui si parler in seguito in questo capitolo. In modo analogo al
sistema di esercizi originale, l'utente deve simulare il comportamento
dell'algoritmo in analisi, ed ha la possibilit di valutare il proprio
risultato, e di vederne la risoluzione corretta.


\section{Modifiche apportate}

E' stato scelto di sviluppare solo esercizi di simulazione di algoritmi
perch in questo modo lo studente  il protagonista effettivo dell'animazione.
Il programma quindi non permette all'utente di realizzare una propria
animazione, in quanto questa funzione del programma MatrixPro non
 stata inclusa nel nuovo software. E' stata effettuata questa scelta
per due motivazioni: 
\begin{itemize}
\item i destinatari del programma sono studenti, non insegnanti, e quindi
si presuppone (in accordo con gli studi riportati nel capitolo \ref{capitolo1})
che siano pi portati a utilizzare e personalizzare le visualizzazioni
piuttosto che a creare nuove animazioni.
\item esistono gi validi strumenti per la creazione di animazioni personalizzate,
come MatrixPro, che permettono di realizzarle semplicemente anche
in pochi minuti; l'obiettivo di questa tesi era di realizzare qualcosa
di nuovo e poco diffuso, piuttosto di creare un programma che  gia
disponibile.
\end{itemize}
Inoltre, se si rendesse necessario,  possibile in un secondo momento
integrare il progetto al sistema MatrixPro in modo analogo a come
il sistema Trakla2  stato unito a quest'ultimo.

{[}parlare della scelta iniziale???{]}

Per realizzare al meglio un'applicazione semplice nell'uso, e soprattutto
interattiva, sono state apportate alcune modifiche ed estenzioni al
programma origianale. Nelle prossime sezioni saranno analizzate in
dettaglio le variazioni.


\subsection{Tipologie di Esercizi}

Durante la realizzazione del progetto, si  osservato che per alcune
categorie di algoritmi la tecnica dell'esercizio di simulazione non
era adatta, in quanto l'esercizio risultava essere poco interessante
per l'utente poich consisteva in un gran numero di operazioni da
eseguire tutte simili tra loro. L'esempio pi rappresentativo di questa
problematica  la categoria degli algortimi di programmazione dinamica:
lo studente dovendo riempire con dei numeri una tabella (vedi esercizio
Longest Common Subsequence), perde interesse nell'argomento e non
approfondisce adeguatamente il problema e la sua soluzione.

Per questo motivo si  scelto di dividere la raccolta di problemi
in due sottoinsiemi: gli esercizi e le animazioni. Per la prima parte
si sono realizzati esercizi di simulazione molto simili a quelli forniti
da Trackla2, mentre per la seconda parte sono stati prodotte animazioni
dell'algoritmo, analoghe alla modalit {}``Model Answer'' degli
esercizi. Per le simulazioni,  stata quindi sacrificata un po' di
interattivit a favore di una maggiore chiarezza dell'esercizio.


\subsection{\label{sub:Input-a-scelta}Input a scelta}

In accordo con i risultati degli studi di cui si parla nel capitolo
\ref{SCORRETTA:Ref: capitolo1},  stata aggiunta al programma la
possibilit per l'utente di inserire dati in ingresso scelti personalmente.
Il software Trakla2 invece automaticamente generava un insieme di
dati casuali.

Questa nuova funzione garantisce un buon livello di interattivit
dell'utente, con il minimo sforzo da parte di quest'ultimo: lo studente
pu infatti inserire particolari dati per osservare il comportamento
dell'algoritmo nei casi limite. E' sempre comunque permesso l'utilizzo
di input casuali generati dal sistema.

La preferenza per una particolare tipologia di input avviene successivamente
alla scelta dell'esercizio: allo studente si presenta una finestra
in cui deve decidere se utilizzare i propri dati oppure farli generare
al programma. Se opta per l'input personalizzato, a seconda dell'esercizio
in analisi utilizzer diversi tecniche per l'inserimento. In particolare
il diverso procedimento dipende dal tipo di struttura dati che utilizza
l'algoritmo. Il primo passo  comune alle tre tecniche: l'utente deve
inserire l'elenco delle chiavi (elementi di un vettore, nodi di un
grafo, nomi dei punti) all'interno di un campo di testo. In alcuni
casi non  previsto l'inserimento di queste chiavi, in quanto non
sono necessarie per la particolare simulazione. La dimensione dell'input
 definita implicitamente in questo passaggio. Il prossimo  invece
discriminante delle varie tecniche, e lo analizziamo nel dettaglio:
\begin{itemize}
\item Esercizi {}``normali'', in questa categoria rientrano tutti gli
esercizi che non utilizzano particolari strutture dati (grafi o aree
geometriche), e quindi principalmente contengono vettori e/o alberi.
Se l'esercitazione necessita solo di un elenco di valori-chiave con
i quali riempire la struttura, allora nessun'altra finestra viene
visualizzata, e lo studente passa direttamente all'attivit di simulazione.
Se sono previsti dati aggiuntivi, come per esempio i parametri di
una funzione di hashing, si presenta una finestra in cui  possibile
inserire i valori di suddetti parametri aggiuntivi.
\item Grafi, la creazione di un grafo necessita di un elenco di nodi, ma
soprattutto della relazione di adiacenza tra questi. Nell'applicazione
l'inserimento di tale informazione  stato realizzato tramite una
matrice di adiacenza. Lo studente si trova di fronte ad una seconda
finestra, in cui  presentata una matrice costruita nel seguente modo:
le cui righe e colonne sono etichettate con i nodi precedentemente
inseriti, ed ogni elemento interno  un'area di testo (se il grafo
ha archi pesati) in cui si deve inserire il peso del relativo arco
(0 equivale a arco assente), oppure una casella di spunta per indicare
se esiste o meno un arco tra i due nodi. Se si sta parlando di un
grafo non orientato,  possibile inserire le informazioni della relazione
solo nella parte triangolare inferiore della matrice.
\item Geometria, per i due problemi di geometria computazionale  stato
creata una tecnica di input molto simile al primo tipo, in cui  possibile
inserire le coppie di coordinate (x,y) di ogni punto precedentemente
aggiunto.
\end{itemize}
Inizialmente era prevista anche un'ulteriore tipologia di inserimento
dati: i Test Cases {[}tradurre{]}. Essi prevedevano di fornire un
paricolare insieme di dati preparati dal programmatore (come suggerito
della guida \cite{wikiAlgoViz}) che permettessero allo studente di
osservare il comportamento dell'algoritmo in situazioni limite, poich
egli potrebbe non avere l'intuizione corretta per realizzarle tramite
un input personalizzato. Successivamente questa possibilit  stata
accantonata per problemi di tempi nel completamento del progetto,
e per concentrarsi su altri aspetti dell'applicazione. L'inserimento
di tale funzione in un secondo momento non  comunque esclusa.


\subsection{Documentazione algoritmo}

Sono state fatte alcune integrazioni anche dal punto di vista dell'interfaccia
grafica del software, in particolare per la finestra di simulazione
degli esercizi, che sono descritte in questa e nelle due sezioni successive.

Per prima cosa  stato aggiunto un pannello laterale nella parte sinistra
della finestra, che contiene a sua volta tre pannelli tab {[}tabpanel..
come si traduce?{]}. In questa sezione analizzeremo due di questi
pannelli.

Il pannello {}``Descrizione'' contiene alcune informazioni dell'algoritmo,
in particolare una breve descrizione del problema affrontato, e le
istruzioni fondamentali per poter affrontare la simulazione nell'esercizio. 

Un secondo pannello contiene lo pseudo codice dell'algoritmo in questione,
ed  tratto dal libro \cite{libro}. In alcuni casi il codice di un
particolare problema non era presente su questo libro, e quindi si
riferisce ad un altro libro \cite{cormen}.

La gestione dei testi relativi ad ogni esercizio  stata realizzata
con l'aiuto di un file di testo esterno al codice, contenente tutte
queste propriet. In questo modo si semplificano eventuali modifiche
future della documentazione, che non necessitano la compilazione del
codice sorgente, ma che consistono semplicemente in un aggiornamento
di questo file. Inoltre, si permette ad un potenziale docente che
utilizza un diverso testo per il proprio corso, di aggiornare i codici
e le informazioni degli algoritmi secondo le proprie preferenze.

La scelta di aggiungere questa sezione informativa al software  stata
fatta in seguito all'analisi degli altri programmi disponibili \ref{SCORRETTA:Ref: sec:Strumenti-Disponibili},
ma che JHAVE' \cite{JHAVE} il quale forniva allo studente una documentazione
sull'argomento come integrazione all'esercizio. Anche qui i testi
di sussidio non erano integrati nel codice, ma memorizzati in pagine
html {[}verificare{]}.


\subsection{\label{sub:Domande-relative-esercizio}Domande relative esercizio}

Un'altra funzionalit che  stata aggiunta al programma prendendo
spunto dalle caratteristiche di JAHVE' \cite{JHAVE},  la possibilit
di confrontarsi con alcune domande a proposito dell'algoritmo o della
struttura dati analizzati. Nel programma JHAVE, l'animazione del problema
viene interrotta ponendo domande sull'argomento allo studente, con
una frequenza che varia a seconda di alcuni parametri impostati dal
programmatore. Lo studente per poter continuare l'animazione deve
rispondere alle domande che gli sono poste. Provando ad utilizzare
questo programma, alcuni studenti del corso di Informatica hanno trovato
fastidioso essere continuamente fermati nella presentazione da queste
domande. Nel nuovo progetto realizzato si  quindi optato per inserire
la possibilit di rispondere alle domande, ma a scelta dello studente.

Il pulsante per poter sottoporsi ad una domanda  posto nel terzo
pannello sulla sinistra, insieme ai due di cui si  parlato nelle
sezioni precedenti. Le domande, che variano da 3 a 5 per ogni esercizio,
sono di due tipologie: a scelta multipla o a risposta aperta (in cui
lo studente risponde scrivendo all'interno di un campo di testo),
e vengono visualizzate tramite una piccola finestra. Una volta inserita
la risposta, il programma comunica allo studente se  esatta o meno;
nel caso fosse sbagliata viene visualizzata la soluzione giusta. Le
domande e le loro risposte sono inserite direttamente nel codice di
ogni esercizio. Questo  sicuramente un punto debole, e quindi un
miglioramento futuro potr trasportarle all'esterno del codice, magari
in un file di testo esterno, come per la documentazione degli esercizi.

E' stata inoltre sperimentata la possibilit di creare domande {}``dinamiche'',
ovvero che interagiscono direttamente con il contenuto dell'esercizio.
Per esempio in un esercizio di inserimento di dati in un albero di
ricerca binario, si potrebbe chiedere: {}``In quale nodo andr inserita
la prossima chiave K'' {}``(Risposta) S''. In questo caso, sia
la domanda che la risposta, saranno modificate con i dati dell'esercizio,
sostituendo a K il valore della chiave che si  prossimi ad inserire
nella struttura, e a S il nome del nodo che diventer padre di K.


\subsection{FeedBak RealTime}

I programmi di animazione di algoritmi sono usati spesso da studenti
alle prime armi con un determinato argomento, e che hanno bisogno
di vedere e toccare con mano per riuscire a focalizzare al meglio
il problema.s Uno studente non ferrato in un particolare algoritmo
non riuscir a simulare correttamente gi la prima volta l'esercizio,
e dovr ricorrere continuamente alla funzione di Model Answer per
controllare se si sta muovendo correttamente nella struttura dati.
Per agevolare il primo apprendimento dell'argomento, si  quindi pensato
di aggiugere, sempre nella parte sinistra dello schermo, un'area di
testo che reagisce ai cambiementi effettuati dall'utente sulle strutture
dati, modificando il proprio testo e comunicando attraverso questo
allo studente se la simulazione di ogni passo  corretta. Per ogni
esercizio il criterio con cui si stabilisce se un particolare passo
 corretto, e il testo che di conseguenza deve apparire, sono diversi
e incorporati nel codice.

Per un utente esperto che vuole verificare le proprie conoscenze,
questa funzione pu essere di disturbo:  stata perci inserita la
possibilit di disattivare l'area di testo semplicemente un pulsante.
L'area pu essere tuttavia riattivata in un secondo momento.


\section{\label{sec:Esempi}Esempi}

descrizione di due esempi uno per simulazione esercizio, l'altro per
animazione. 


\subsection{esercizio ABR insert}

descrizione esercizio abr insert


\subsection{Esercizio Fibonacci o Graham}

descrizione esercizio Fibonacci

\begin{figure}[htbp]
\centering
\includegraphics[scale=0.4]{images/StateChart.png}
\caption{Diagramma utilizzo dello studente}
\end{figure} 


\section{Conlusioni capitolo e problemi riscontrati}

Lo scopo finale di questo lavoro era la realizzazione di un programma
che permettesse allo studente di Algoritmi e Struture Dati di avere
un supporto al corso pi interattivo rispetto ai programmi gi disponibili.
In quanto programma sperimentale, il software risultante  ricco di
nuove funzioni che aumentano la partecipazione dell'utente nell'attivit,
ma che purtroppo non sono ancora perfette.

Come si  gi accennato in questo capitolo, le migliorie introdotte
possono essere ulteriormente perfezionate, per fornire agli studenti
un valido aiuto per lo studio. Nonostante non siano state ancora realizzate
nel progetto, sono gi state concepite pianificate alcune modifiche:
\begin{description}
\item [{Domande}] Innanzi tutto si potrebbe aumentare la quantit di domande
per ogni esercizio, e migliorare il sistema di inserimento delle risposte:
ad oggi il sistema giudica diverse le riposte {}``O(nlogn)'' e {}``O(n
logn)''. Come  gi stato accennato in \ref{sub:Domande-relative-esercizio},
sarebbe utile spostare le domande in un file di testo esterno, di
modo che non sia necessario ri-compilare tutto il codice del programma
nel caso si modifichi solo una domanda, e per permettere ad un docente
interessato di personalizzare le domande in accordo con il proprio
programma di studio.
\item [{Input1}] Nella sezione \ref{sub:Input-a-scelta}, si parla della
possibilit si fornire una terza tipologia di input, i {}``Test Cases'',
ovvero un insieme di dati preparati ad hoc per ogni algoritmo, per
evidenziare particolari comportamenti e caratteristiche del problema
in analisi. Sicuramente questa potrebbe essere una delle prime integrazioni
al programma della tesi. Un ulteriore miglioria possibile attuabile,
poterbbe consistere nel permettere di memorizzare in un file di testo
i test case per ogni esercizio, che sarebbero modificabili da un docente,
come per domande e documentazione
\item [{Input2}] La tecnica di inserimento dei dati potrebbe inoltre essere
integrata con il sistema di creazione di animazioni {}``al volo''
di MatrixPro. Il software, come spiegato in \ref{sec:MatrixPro-e-Trackla2},
permette di inserire e personalizzare strutture dati preinstallate
nell'applicazione per poter creare animazioni. Sarebbe un grande passo
avanti permettere agli utenti del programma di inserire i dati di
input di un esercizio con lo stesso sistema di creazione della animazioni
di MatrixPro: sicuramente renderebbe gli studenti pi fantasiosi nel
creare casi limite, soprattutto con strutture grafiche come grafi
e alberi.
\item [{Esercizi}] Il programma di un corso di Algoritmi  sicuramente
vasto, e per ogni argomento una dimostrazione non basta di sicuro.
Potrebbe essere necessario in futuro allargare la raccolta di esercizi
per fornire agli studenti un percorso ricco e completo di supporto
al proprio corso di studi. 
\end{description}
