
\chapter{Implementazione} 

In questo capitolo si descrivono e si motivano le scelte progettuali
realizzate nell'implementazione dell'applicazione, soffermandosi in
particolar modo sulle migliorie apportate al programma di partenza.
Sono inoltre descritti due esercizi rappresentativi \ref{sec:Esempi},
e una breve spiegazione sull'utilizzo da parte dello studente\ref{sec:Come-si-usa}.


\section{Descrizione iniziale}

Il software realizzato si presenta allo studente con una schermata
iniziale contenente l'elenco degli esercizi preparati organizzati
tramite una struttura ad albero per argomenti. Selezionando un particolare
esercizio si visualizza nella parte inferiore della finestra una breve
descrizione dell'esercitazione.

Una volta scelta l'attivit, allo studente si apre un'ulteriore finestra,
molto simile a quella del sistema Trakla2, ma con qualche integrazione,
di cui si parler in seguito in questo capitolo. In modo analogo al
sistema di esercizi originale, l'utente deve simulare il comportamento
dell'algoritmo in analisi, ed ha la possibilit di valutare il proprio
risultato, e di vederne la risoluzione corretta.


\section{Modifiche apportate}

E' stato scelto di sviluppare solo esercizi di simulazione di algoritmi
perch in questo modo lo studente  il protagonista effettivo dell'animazione.
Il programma quindi non permette all'utente di realizzare una propria
animazione, in quanto questa funzione del programma MatrixPro non
 stata inclusa nel nuovo software. E' stata effettuata questa scelta
per due motivazioni: 
\begin{itemize}
\item i destinatari del programma sono studenti, non insegnanti, e quindi
si presuppone (in accordo con gli studi riportati nel capitolo \ref{capitolo1})
che siano pi portati a utilizzare e personalizzare le visualizzazioni
piuttosto che a creare nuove animazioni.
\item esistono gi validi strumenti per la creazione di animazioni personalizzate,
come MatrixPro, che permettono di realizzarle semplicemente anche
in pochi minuti; l'obiettivo di questa tesi era di realizzare qualcosa
di nuovo e poco diffuso, piuttosto di creare un programma che  gia
disponibile.
\end{itemize}
Inoltre, se si rendesse necessario,  possibile in un secondo momento
integrare il progetto al sistema MatrixPro in modo analogo a come
il sistema Trakla2  stato unito a quest'ultimo.

{[}parlare della scelta iniziale???{]}

Per realizzare al meglio un'applicazione semplice nell'uso, e soprattutto
interattiva, sono state apportate alcune modifiche ed estenzioni al
programma origianale. Nelle prossime sezioni saranno analizzate in
dettaglio le variazioni.


\subsection{Tipologie di Esercizi}

Durante la realizzazione del progetto, si  osservato che per alcune
categorie di algoritmi la tecnica dell'esercizio di simulazione non
era adatta, in quanto l'esercizio risultava essere poco interessante
per l'utente poich consisteva in un gran numero di operazioni da
eseguire tutte simili tra loro. L'esempio pi rappresentativo di questa
problematica  la categoria degli algortimi di programmazione dinamica:
lo studente dovendo riempire con dei numeri una tabella (vedi esercizio
Longest Common Subsequence), perde interesse nell'argomento e non
approfondisce adeguatamente il problema e la sua soluzione.

Per questo motivo si  scelto di dividere la raccolta di attivit
in due sottoinsiemi: gli esercizi e le simulazioni. Per la prima parte
si sono realizzati esercizi di simulazione 


\subsection{Input a scelta}


\subsection{Documentazione algoritmo laterale}


\subsection{Domande relative esercizio}


\subsection{testo di aiuto}


\section{\label{sec:Come-si-usa}Come si usa}


\section{\label{sec:Esempi}Esempi}


\section{Implementazione}

uno schema veloce delle classi, magari anche un activity con le operazioni
base che pu fare lo studente


\section{Conlusioni capitolo e problemi riscontrati}
