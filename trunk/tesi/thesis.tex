%% LyX 1.6.4 created this file.  For more info, see http://www.lyx.org/.
%% Do not edit unless you really know what you are doing.
\documentclass[12pt,a4paper,twoside,openright]{book} 
\usepackage[small,anne]{caption2}  
\usepackage[utf8x]{inputenc}
\usepackage[english]{babel}
\usepackage[T1]{fontenc}
\usepackage{textcomp}
\usepackage{float}
\usepackage{fancyheadings}
\usepackage{fancybox}
\usepackage{setspace}
\usepackage{frontesp} % first page
\usepackage{paralist}

\usepackage{graphicx}
\usepackage{rotating}

% From TCAD paper
\usepackage{cite}
\usepackage{times}
\usepackage[OT1]{fontenc}
\usepackage{type1cm}
\usepackage{url}
\usepackage{subfigure}
\usepackage{amssymb}
\usepackage{amsmath}
\usepackage{graphicx}
\usepackage{xspace}
%\usepackage[ruled,linesnumbered]{algorithm2e}
% from tcad paper
%%\usepackage[Lenny]{sty/fncychapleo}
\usepackage[Bjornstrup]{sty/fncychapleo}
\usepackage{tikz}
% Optional PGF libraries
\usepackage{pgflibraryarrows}
\usepackage{pgflibrarysnakes}
\usepackage{indentfirst}

% test
\usepackage{hyperref}

\begin{document} 
%% dimensione interlinea
\setstretch{1.5}
%\linespread{1.3}

% print first page
% commands used by the first page
% titolo
\title{Percorso Interattivo per l'Apprendimento degli Algoritmi e Strutture Dati}
% candidato
\providecommand{\autore}{\Large{Stella Margonar}}
% relatore
\providecommand{\principaladviser}{\Large{prof. Alberto Montresor}} 
% correlatore
\providecommand{\firstreader}{\Large{Primo correlatore}}
\providecommand{\secondreader}{\Large{Secondo correlatore}}
\providecommand{\annoacc}{\Large\textbf{2009 - 2010}}

% Print First page
\titlep
% end of management of first page

% Introductive part
\frontmatter
\pagenumbering{roman}
% % -*-latex-*-

\chapter*{Ringraziamenti}
\setstretch{2.2}
\vspace{3.3cm}
\begin{flushright}
\textit{A pippo} 
\end{flushright}

\begin{flushright}
\indent\textit{A pluto}
\end{flushright}

\begin{flushright}
\indent\textit{A bambi}
\end{flushright}
\setstretch{1.5}

\tableofcontents
\listoffigures
\listoftables
\clearpage



% Main part
\mainmatter
\setlength{\evensidemargin}{0cm}
\setlength{\oddsidemargin}{+1.5cm}
\setlength{\parindent}{1.4cm}
\pagestyle{fancy}
\lhead[{\bfseries {\em \leftmark}}]{}
\rhead[]{{\it \rightmark}}
\renewcommand{\chaptermark}[1]{\markboth{#1}{}}
\renewcommand{\sectionmark}[1]{\markright{#1}{}}

\chapter{Stato dell'arte}


\section{Introduzione}

boh..


\section{Animazione e Simulazione}

A partire dagli anni '80, lo sviluppo animazioni per la visualizzazione
di algoritmi  stata in continuo aumento. La prima testimonianza che
si ha  {}``Sorting out sorting'', un video di presentazione degli
algoritmi di ordinamento, primo nel suo genere.

Dal 1981 fino ad oggi, le rappresentazioni di algoritmi e strutture
dati si sono diffuse ed hanno trovato il loro scopo finale come supporto
per studenti ed insegnanti nei relativi corsi universitari.

Osservando il materiale disponibile ad oggi sul web e come ci fa notare
{[}1{]}, possiamo riconoscere che esistono principalmente due forme
in cui si sono sviluppati i tool di algorithm visualization: l'animazione
e la simulazione.

La prima consiste in una visualizzazione degli effetti che un algoritmo
predefinito ha su una struttura dati definita dal programmatore. L'utente,
in questo caso, assiste passivamente all'animazione osservandone i
risultati sulla struttura dati.

Nel secondo caso parliamo di simulazione in quanto l'utente manipola
gli oggetti messi a disposizione dal programmatore secondo le modifiche
permesse dalla struttura in modo da creare una sequenza di passi analoghi
a quelli dell'algoritmo che si sta analizzando.

Naturalmente questa  una classificazione grossolana, in quanto ogni
tool ha le sue particolari caratteristiche. Le funzionalit introdotte
negli anni nelle varie applicazioni disponibili hanno permesso una
seconda classificazione pi accurata in base all'interazione che tali
visualizzazioni hanno con gli studenti. Il risultato di gruppo di
lavoro su {}``Improving the Educational Impact of Algorithm Visualization''
(Giugno 2002, ITiCSE, Denmark)  stata la suddivisione in 6 categorie
delle varie forme di insegnamento in base al supporto di visualizzazione
utilizzato, riportata in {[}3{]}:
\begin{enumerate}
\item No viewing (come lo traduco??), gli studenti non hanno a disposizione
nessun supporto visivo, (o al massimo immagini non animate);proprio 
\item Viewing, osservazione passiva dell'esecuzione dell'algoritmo, in cui
 possibile al massimo controllare il flusso di esecuzione tramite
comandi di undo - redo - rewind....
\item Responding, come per viewing consiste nell'osservazione della rappresentazione
dell'algoritmo, durante la quale allo studente  richiesto di rispondere
ad una serie di domande, o svolgere esercizi su carta, relativi all'argomento
presentato.
\item Changing, modificare la visualizzazione principalmente attraverso
la personalizzazione dei dati in ingresso all'algoritmo, in modo da
analizzare il comportamento dell'algoritmo nei vari casi.
\item Constructing, costruire la propria visualizzazione dell'algoritmo
che si sta studiando
\item Presenting, presentare ad un pubblico la visualizzazione di un particolare
algoritmo, per poi procedere ad una discussione sull'argomento.
\end{enumerate}
I risultati delle tipologie di insegnamento appena riportate sono
soggetto di numerosi studi presenti e passati, e una sinteri dei risulati
raccolti  presente nella \ref{sec:studi-effettuati-sull'efficacia}


\section{\label{sec:studi-effettuati-sull'efficacia}Studi effettuati sull'efficacia
degli AV nell'insegnamento}

A seguito della classificazione delle tipologie di insegnamento dei
corsi riguardanti Algoritmi e Strutture Dati, e la loro interazione
con le tecnologie di visualizzazione algoritmi, sono stati effettuati
numerosi studi a proposito dell'efficacia dell'introduzione di queste
nuove tecniche sull'apprendimento degli studenti.

Un'analisi generale  riportata in {[}3{]}. Lo studio {[}7{]} hanno
riportato gli studenti analizzati che hanno utilizzato un supporto
alle lezioni di tipo 3(responding) e 5(construction), hanno ottenuto
risultati migliori nei test, rispetto ad altri studenti che non ne
avevano fatto uso, o che avevano utilizzato una semplice animazione
(livello 2). Negli studi {[}9,10{]} sono stati raccolti invece risultati
in conflitto con i primi. Secondo questi ultimi, non c' stata nessuna
differenza significati nei risultati ottenuti dagli studenti con l'una
o l'altra tipologia di supporto alle lezioni.

Lo studio riportato in {[}3{]} mette in risalto il miglioramento degli
studenti, a seconda della tecnica di supporto visivo utilizzata, tenendo
anche conto anche delle conoscenze precedenti dei soggetti. Gli studenti,
provenienti da 3 diverse universit, sono stati soggetti di un pre-test,
nel quale venivano valutate le loro conoscenze e attitudini, prima
delle lezioni. A seconda dei risultati, sono stati divisi in tre gruppi
per ognuno dei quali sono state effettuate una serie di lezioni sull'argomento
degli algoritmi di ordinamento con l'utilizzo di 3 diverse tecniche:
\begin{enumerate}
\item Lezioni frontali senza supporti visivi
\item Lezioni frontali con l'utilizzo da parte del docente di una presentazione
e di un tool di visualizzazione di algoritmi
\item Lezioni frontali (come gruppo 2) con l'aggiunta di ore supplementari
di attivit di laboratorio, in cui gli studenti potevano utilizzare
singolarmente un'applicazione (JHAVE' {[}??{]}) che permette di osservare
l'animazione degli algoritmi integrata con domande ad-hoc.
\end{enumerate}
A seguito di queste attivit, gli studenti sono stati esaminati con
un test uguale per tutti.

I risultati ottenuti si sono rilevati concordi con {[}7,8{]} in quanto
rapportando le conoscenze iniziali degli studenti con gli esiti dei
test di verifica, si  osservato che maggiore  l'interazione con
lo strumento di supporto, maggiori sono i progressi ottenuti dagli
studenti.

A seguito di questo studio si  quindi potuto supporre che i risultati
non significativi di {[}9{]} fossero dovuti ad un'inadeguata tecnica
di raccolta: nel caso delle domande gli studenti, lasciati liberi
a se stessi e non confinati in un laboratorio, hanno affrontato il
tool come un gioco e non come uno strumento educativo.

Per quanto riguarda la categoria di applicazioni 6 (presentig), purtroppo
non sono ancora presenti studi che indichino l'efficacia di tale tecnica
di insegnamento.

{[}Aggiungere testimonianza corso di studi algoritmie strutture dati
in cui sono state adoperate le tecniche sopra, obbligando gli studenti,
che le hanno trovate utili e il 65\% ha passato l'esame al primo colpo{]}


\section{Come creare un buon AV}

Algoviz {[}5{]} costituisce il sito di riferimento per studenti docenti
e chiunque altro sia interessato nella ricerca di un'applicazione
che raccolga visualizzazioni di algoritmi, o l'animazione di una particolare
struttura. Dal 2006 il sito raccoglie in un proprio archivio (continuamente
aggiornato) tutte gli strumenti rintracciabili sul web relativi a
questo argomento. Tale sito  stato anche il punto di partenza di
questa tesi, in quanto ha permesso di esplorare e analizzare i principali
toolkit per sviluppatori presenti al momento, e inoltre contiene un'interessante
guida per chi si appresta a realizzare una propria visualizzazione
di algoritmi. 

Si riporta ora una sintesi dei principi chiave esposti nella guida:
\begin{description}
\item [{Coinvolgere}] fare in modo che l'utente partecipi attivamente durante
l'animazione dell'algoritmo, evitando le presentazioni statiche e
passive
\item [{Semplificare}] ridurre la complessit dell'applicazione, a partire
dall'interfaccia grafica. Per quanto riguarda l'esposizione dei concetti,
nascondere per quanto possibile i dettagli implementativi degli algoritmi,
e animare le i singoli passi operazioni pi complesse. L'applicazione
inoltre deve essere talmente semplice da necessitare di una minima
documentazione relativa al proprio funzionamento, in quanto l'utente
deve riuscire ad avviarla e comprenderne l'utilizzo fin dal primo
avvio.
\item [{DataSet}] fornire diverse tipologie di gestione dei dati in ingresso

\begin{itemize}
\item Dati inseriti dall'utente, per permettere l'eplorazione di tutte le
possibili configurazioni, dando la possibilit allo studente di verificare
il comportamento della struttura in casi particolari da lui costruiti;
\item Dati random, di cui l'utente potrebbe controllare solo alcuni parametri
(con la dimensione, il range, ecc..);
\item Dati pre-costruiti ad hoc, che permettano allo studente di analizzare
il comportamento chiave dell'algortimo in situazioni chiave, preparate
ad hoc dal docente o dal programmatore. Questa funzionalit risulta
utile perch generalmente gli studenti (soprattutto se alle prime
armi) non ha la fantasia necessaria per creare casi particolari di
testing. 
\end{itemize}
\item [{Controllo}] se si tratta di una semplice rappresentazione animata
dell'esecuzione di un programma su di una struttura dati,  fondamentale
permettere allo studente di controllare il flusso dell'animazione
attraverso una serie di comandi, quali il controllo della velocit
(nel caso si parli di un'animazione continua), possibilit di procedere
passo-passo, andare avanti e indietro nell'animazione. Algoviz riporta
che l'utilizzo dell'animazione spezzata in passi, produce risultati
migliori nell'apprendimento rispetto ad una visualizzazione continua
del funzionamento.
\end{description}

\section{\label{sec:Strumenti-Disponibili}Strumenti Disponibili}

La relazione {[}2{]} redatta dal gruppo di lavoro di Algoviz {[}\textonesuperior{}{]},
illustra il lavoro di raccolta di AV da essi realizzato nel 2006:
sono stati trovati circa 350 AV all'epoca, che sono quindi stati raccolti
e classificati nel catalogo presente nel sito, il quale  settimanalmente
aggiornato e ad oggi contiene pi di 500 visualizzazioni. Gran parte
degli strumentu catalogati riguarda gli argomenti principali dell'insegnamento
di Algoritmi e Strutture dati, primo fra tutti il problema dell'ordinamento,
al quale seguono le strutture di ricerca (ABR) strutture lineari e
gli algoritmi sui grafi.

A partire da questa raccolta, si sono analizzati i principali pacchetti
che potenzialmente costituivano un punto di partenza per altri sviluppatori
dal quale partire per creare la propria applicazione di animazione
di algoritmi. Gran parte delle applicazioni raccolte consistono in
passive rappresentazioni di algortimi che agiscono su strutture dati
preinstallate (quindi non personalizzabili) nelle quali l'utente 
coinvolto solo per quanto riguarda il controllo del flusso. Esistono
per alcune eccezioni, in cui tale rappresentazione  arricchita con
funzionalit interessanti che catturano e focalizzano meglio l'attenzione
dello studente sull'argomento.

Di seguito sono descritti alcuni dei principali pacchetti che sono
stati analizzati durante la ricerca. Molti altri software sono stati
esaminati, ma in questa circostanza non vengono riportati, in quanto
sono risultati presentare caratteristiche molto simili a quelle dei
programmi che ora verranno esposti.
\begin{description}
\item [{Animal}] Il software Animal {[}11{]} fornisce 
\item [{JHAVE'}] pacchetto migliore, classificato al 2 posto, in quanto
non si capice come modificarlo e le domande fanno cagare
\item [{MatrixPro}] e Trackla2, vedi seguito
\end{description}
\chapter{Implementazione}


\section{MatrixPro e Trackla2}

MatrixPro {[}1{]}  un software con licenza GPL2, ed  quindi gratuito
e open source. Il tool  stato creato per permettere a studenti, ma
soprattutto a insegnanti di creare velocemente animazioni personalizzate
su strutture dati preesistenti nel programma. L'applicazione  molto
utile come spiegano gli sviluppatori nell'articolo {[}1{]} per rendere
pi coinvolgenti le lezioni universitarie, con presentazioni animate
precedentemente create dal docente, o addirittura create al volo durante
la lezione, per riuscire a rispondere con la dimostrazione visiva
a domande e dubbi degli studenti.

Tra i punti di forza di questo applicativo troviamo la semplicit
con cui possono essere utilizzate le strutture dati esistenti, ma
soprattutto l'organizzazione di queste. Le strutture sono suddivise
in due grandi gruppi: 
\begin{description}
\item [{FDT}] (Fundamental Data Types) che equivalgono alle strutture basilari,
primitive alle quali non viene associata alcuna operazione o informazione
sematica riguardo al loro utilizzo. Tra queste troviamo liste, alberi,
grafi, array, chiavi.
\item [{CDT}] (Conceptual Data Types) sono strutture che implementano tipi
di dato astratti che necessitano di maggiori restrizioni, e possono
essere modificate solo tramite operazioni che ne modificano la struttura
secondo regole predefinite. Tra queste troviamo Alberi Binari di Ricerca,
Heap, Stack, e molte altre.
\end{description}
Esistono inoltre altre strutture utilizzabili nella realizzazione
di un'animazione, che vengono chiamate {}``strutture utili'', e
permettono di creare al volo un input random o consentono altre funzionalit
allo stesso modo interessanti. Questa classificazione in strutture
primitive e avanzate aiuta l'utente a comprendere meglio la gerachia
esistente tra i diversi sistemi, e a concentrarsi maggiormente sulla
parte concettuale del problema analizzato e non sui dettagli implementativi.


\section{modifiche apportate}


\subsection{Esercizi + anche solo simulazione}


\subsection{Input a scelta}


\subsection{Documentazione algoritmo laterale}


\subsection{Domande relative esercizio}


\subsection{testo di aiuto}


\section{Esempi}


\section{Implementazione}

uno schema veloce delle classi, magari anche un activity con le operazioni
base che pu fare lo studente

\chapter{Implementazione} 

In questo capitolo si descrivono e si motivano le scelte progettuali
realizzate nell'implementazione dell'applicazione, soffermandosi in
particolar modo sulle migliorie apportate al programma di partenza.
Sono inoltre descritti due esercizi rappresentativi \ref{sec:Esempi},
e una breve spiegazione sull'utilizzo da parte dello studente\ref{sec:Come-si-usa}.


\section{Descrizione iniziale}

Il software realizzato si presenta allo studente con una schermata
iniziale contenente l'elenco degli esercizi preparati organizzati
tramite una struttura ad albero per argomenti. Selezionando un particolare
esercizio si visualizza nella parte inferiore della finestra una breve
descrizione dell'esercitazione.

Una volta scelta l'attivit, allo studente si apre un'ulteriore finestra,
molto simile a quella del sistema Trakla2, ma con qualche integrazione,
di cui si parler in seguito in questo capitolo. In modo analogo al
sistema di esercizi originale, l'utente deve simulare il comportamento
dell'algoritmo in analisi, ed ha la possibilit di valutare il proprio
risultato, e di vederne la risoluzione corretta.


\section{Modifiche apportate}

E' stato scelto di sviluppare solo esercizi di simulazione di algoritmi
perch in questo modo lo studente  il protagonista effettivo dell'animazione.
Il programma quindi non permette all'utente di realizzare una propria
animazione, in quanto questa funzione del programma MatrixPro non
 stata inclusa nel nuovo software. E' stata effettuata questa scelta
per due motivazioni: 
\begin{itemize}
\item i destinatari del programma sono studenti, non insegnanti, e quindi
si presuppone (in accordo con gli studi riportati nel capitolo \ref{capitolo1})
che siano pi portati a utilizzare e personalizzare le visualizzazioni
piuttosto che a creare nuove animazioni.
\item esistono gi validi strumenti per la creazione di animazioni personalizzate,
come MatrixPro, che permettono di realizzarle semplicemente anche
in pochi minuti; l'obiettivo di questa tesi era di realizzare qualcosa
di nuovo e poco diffuso, piuttosto di creare un programma che  gia
disponibile.
\end{itemize}
Inoltre, se si rendesse necessario,  possibile in un secondo momento
integrare il progetto al sistema MatrixPro in modo analogo a come
il sistema Trakla2  stato unito a quest'ultimo.

{[}parlare della scelta iniziale???{]}

Per realizzare al meglio un'applicazione semplice nell'uso, e soprattutto
interattiva, sono state apportate alcune modifiche ed estenzioni al
programma origianale. Nelle prossime sezioni saranno analizzate in
dettaglio le variazioni.


\subsection{Tipologie di Esercizi}

Durante la realizzazione del progetto, si  osservato che per alcune
categorie di algoritmi la tecnica dell'esercizio di simulazione non
era adatta, in quanto l'esercizio risultava essere poco interessante
per l'utente poich consisteva in un gran numero di operazioni da
eseguire tutte simili tra loro. L'esempio pi rappresentativo di questa
problematica  la categoria degli algortimi di programmazione dinamica:
lo studente dovendo riempire con dei numeri una tabella (vedi esercizio
Longest Common Subsequence), perde interesse nell'argomento e non
approfondisce adeguatamente il problema e la sua soluzione.

Per questo motivo si  scelto di dividere la raccolta di attivit
in due sottoinsiemi: gli esercizi e le simulazioni. Per la prima parte
si sono realizzati esercizi di simulazione 


\subsection{Input a scelta}


\subsection{Documentazione algoritmo laterale}


\subsection{Domande relative esercizio}


\subsection{testo di aiuto}


\section{\label{sec:Come-si-usa}Come si usa}


\section{\label{sec:Esempi}Esempi}


\section{Implementazione}

uno schema veloce delle classi, magari anche un activity con le operazioni
base che pu fare lo studente


\section{Conlusioni capitolo e problemi riscontrati}

\chapter{Conclusioni}


\section{Sviluppi Futuri}


\section{Conclusioni}


\section{Ringraziamenti}
Desidero innanzitutto ringraziare il Professor… per i preziosi insegnamenti durante i due anni di laurea magistrale e le per le numerose ore dedicate alla mia tesi. Inoltre, ringrazio sentitamente il Dr… ed il Dr… che sono stati sempre disponibili a dirimere i miei dubbi durante la stesura di questo lavoro. Intendo poi ringraziare l’Ambasciata…, sottolineando la particolare disponibilità del Dr…, e l’Istituto… per avermi fornito testi e dati indispensabili per la realizzazione della tesi. Inoltre, vorrei esprimere la mia sincera gratitudine al Dr… ed ai miei compagni di corso, in particolare Andrea, Giuseppe e Chiara per i numerosi consigli durante la ricerca. Infine, ho desiderio di ringraziare con affetto i miei genitori per il sostegno ed il grande aiuto che mi hanno dato ed in particolare… per essermi stato vicino ogni momento durante questo anno di lavoro.


\newpage
%\begin{thebibliography}{11}
\bibitem{key-3}Villa Karavirta, Ari Korhonen, Lauri Malmi, and Kimmo
Stalancke in {}``Articolo 26 tool..'' - MatrixPro - A Tool for On-The-Fly
Demonstration of Data Structures and Algorithms

\bibitem{key-1}???? - Algorithm Visualization: A Report on the State
of the Field

\bibitem{key-2} ??? - Algorithm Visualization in CS Education: Comparing
Levels of Student Engagement

\bibitem{key-3} ??? - JAVE' -- An Enviroment to Actively Engage Students
in Web-based Algorithm Visualizations

\bibitem{key-4} http://wiki.algoviz.org/AlgovizWiki/ImplementorsGuide

\bibitem{key-5} Ronald Baeker{}``Sorting out Sorting'' - http://www.youtube.com/watch?v=F3oKjPT5Khg 

\bibitem{key-6} Byrne, M.D. Catrambone, R. and Stasko - Evaluating
Animations as Student Aids in Learning pp 253-278

\bibitem{key-7} stasko 

\bibitem{key-8} Jarc, D. J., Feldman, M.B. and Heller - .......

\bibitem{key-9} HundHausen, C.D. Douglas, S.A., and Stasko - {}``Journal
of Visual Languages and Computing'' pp259-290

\bibitem{key-1} http://www.algoanim.info/Animal2/ Animal2

\bibitem{key-1} Libro montresor che non ricordo il titolo

\bibitem{key-1} Cormen
\end{thebibliography}
\bibliography{thesis}{}
\bibliographystyle{plain}
\end{document}

